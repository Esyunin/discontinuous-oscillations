\documentclass[a4paper,14pt]{extarticle}

\usepackage{cmap}
\usepackage[T2A]{fontenc}
\usepackage[utf8x]{inputenc}
\usepackage[english, russian]{babel}

\usepackage{misccorr} % в заголовках появляется точка, но при ссылке на них ее нет
\usepackage{amssymb,amsfonts,amsmath,amsthm}  
\usepackage{indentfirst}
\usepackage[usenames,dvipsnames]{color} 
\usepackage[unicode,hidelinks]{hyperref}
% \hypersetup{%
%     pdfborder = {0 0 0}
% }

\usepackage{makecell,multirow} 
\usepackage{ulem}
\usepackage{graphicx,wrapfig}
\graphicspath{{img/}}
\usepackage{geometry}
\geometry{left=2cm,right=2cm,top=3cm,bottom=3cm,bindingoffset=0cm,headheight=15pt}
\usepackage{fancyhdr} 
\linespread{1.05} 
\frenchspacing 
\renewcommand{\labelenumii}{\theenumii)} 
\newcommand{\mean}[1]{\langle#1\rangle}
% \usepackage{caption}
%%%%%%%%%%%%%%%%%%%%%%%%%%%%%%%%%%%%%%%%%%%%%%%%%%%%%%%%%%%%%%%%%%%%%%%%%%%%%%%
%%%%%%%%%%%%%%%%%%%%%%%%%%%%%%%%%%%%%%%%%%%%%%%%%%%%%%%%%%%%%%%%%%%%%%%%%%%%%%%

\def\labauthors{Есюнин Д.В., Есюнин М.В.}
\def\labgroup{430}
% \def\department{Кафедра электроники и квантовой физики}
\def\labnumber{1}
\def\labtheme{Исследование динамики систем с разрывными колебаниями}

%%%%%%%%%%%%%%%%%%%%%%%%%%%%%%%%%%%%%%%%%%%%%%%%%%%%%%%%%%%%%%%%%%%%%%%%%%%%%%%
	%применим колонтитул к стилю страницы
\pagestyle{fancy} 
	%очистим "шапку" страницы
\fancyhead{} 
	%слева сверху на четных и справа на нечетных
\fancyhead[L]{\labauthors} 
	%справа сверху на четных и слева на нечетных
%\fancyhead[R]{Отчёт по лабораторной работе №\labnumber} 
	%очистим "подвал" страницы
\fancyfoot{} 
	% номер страницы в нижнем колинтуле в центре
\fancyfoot[C]{\thepage} 
\renewcommand{\phi}{\varphi}
%%%%%%%%%%%%%%%%%%%%%%%%%%%%%%%%%%%%%%%%%%%%%%%%%%%%%%%%%%%%%%%%%%%%%%%%%%%%%%%

\usepackage{float}
\usepackage[mode=buildnew]{standalone}
\usepackage{tikz} 
% \usepackage{subcaption}
\usepackage{tikz,csvsimple}
\usetikzlibrary{scopes}
\usetikzlibrary{%
     decorations.pathreplacing,%
     decorations.pathmorphing,%
    patterns,%
    calc,%
    scopes,%
    arrows,%
    % arrows.spaced,%
}
\makeatletter
\newif\if@gather@prefix 
\preto\place@tag@gather{% 
  \if@gather@prefix\iftagsleft@ 
    \kern-\gdisplaywidth@ 
    \rlap{\gather@prefix}% 
    \kern\gdisplaywidth@ 
  \fi\fi 
} 
\appto\place@tag@gather{% 
  \if@gather@prefix\iftagsleft@\else 
    \kern-\displaywidth 
    \rlap{\gather@prefix}% 
    \kern\displaywidth 
  \fi\fi 
  \global\@gather@prefixfalse 
} 
\preto\place@tag{% 
  \if@gather@prefix\iftagsleft@ 
    \kern-\gdisplaywidth@ 
    \rlap{\gather@prefix}% 
    \kern\displaywidth@ 
  \fi\fi 
} 
\appto\place@tag{% 
  \if@gather@prefix\iftagsleft@\else 
    \kern-\displaywidth 
    \rlap{\gather@prefix}% 
    \kern\displaywidth 
  \fi\fi 
  \global\@gather@prefixfalse 
} 
\newcommand*{\beforetext}[1]{% 
  \ifmeasuring@\else
  \gdef\gather@prefix{#1}% 
  \global\@gather@prefixtrue 
  \fi
} 
\makeatother

\usepackage{booktabs}
\usepackage{pgfplots, pgfplotstable}

\usepackage[outline]{contour}
\usepackage{tocloft}
\renewcommand{\cftsecleader}{\cftdotfill{\cftdotsep}} % for parts
% \renewcommand{\cftchapleader}{\cftdotfill{\cftdotsep}} % for chapters
\usepackage{pgfplots,pgfplotstable,booktabs,colortbl}
\pgfplotsset{compat=newest}
\usepackage{physics}
\usepackage{mathtools}
\mathtoolsset{showonlyrefs=true}
\newcommand\Smat{\hat { \mathbf { S } }}

\newcommand*\dotvec[1][1,1]{\crossproducttemp#1\relax}
\def\crossproducttemp#1,#2\relax{{\qty[\vec{#1}\times\vec{#2}\,]}}

\newcommand*\prodvec[1][1,1]{\crossproducttempa#1\relax}
\def\crossproducttempa#1,#2\relax{{\qty[{#1}\times{#2}\,]}}

% \def\E{\mathscr{E}_H}
\def\Rdim{\,\frac{\text{м}^3}{\text{А} \cdot \text{с}}}

\renewcommand{\vec}{\mathbf} % for parts


\begin{document}
	\begin{titlepage}

		\begin{center}
			
			
			\textsc{Нижегородский государственный университет имени Н.\,И. Лобачевского}
			\vskip 4pt \hrule \vskip 8pt
			\textsc{Радиофизический факультет}
			
			\vfill
			
			{\Large\labtheme}
			
		\end{center}
		
		\vfill
		
		\begin{flushright}
			{Работу выполнили студенты\\ \labauthors\\ 430 группы}
		\end{flushright}
		
		\vfill
		
		\begin{center}
			Нижний Новгород, \the\year
		\end{center}
	\end{titlepage}
%	\tableofcontents
	\newpage 

\section*{Практическая часть}
\subsection*{Измерение периода и амплитуды колебаний мультивибратора}
\begin{figure}[H]
	\centering
	\includegraphics[width=0.7\linewidth]{photo/IMG_3234}
	\caption{Осциллограммы колебаний напряжения и тока}
\end{figure}
Измерили амплитуду и период колебаний напряжения мультивибратора. Амплитуда $A=1,04\:\text{В}$, период $A=56\:\text{мкс}$.
Сфотографировали фазовую плоскость мультивибратора с осциллографа.
\begin{figure}[H]
	\centering
	\includegraphics[width=0.7\linewidth]{photo/IMG_3240}
	\caption{фазовая плоскость мультивибратора}
\end{figure}
\subsection*{Режим триггера}
Перевели схему в режим триггера, измерили длительность снимаемого импульса $\tau = 36\text{мкс}$.
Измерили минимальные и максимальные значения длительности запускающих импульсов, когда схема работает как триггер и переключает состояния $\tau_{max} = ?\:\text{мкс}$, $\tau_{min} = 11,5\:\text{мкс}$.
\begin{figure}[H]
	\centering
	\includegraphics[width=0.6\linewidth]{photo/IMG_3244}
%	\caption{осциллограмма режим триггера}
\end{figure}
\subsubsection*{Осциллограммы напряжения на выходе триггера при делении частоты на триггере}
\begin{figure}[H]
	\centering
	\includegraphics[width=0.6\linewidth]{photo/IMG_3248}
\end{figure}
\subsection*{Режим кипп - реле}
Для схемы кипп-реле измерили длительность выходного сигнала $\tau = ?\:\text{мкс}$. Измерили минимальные и максимальные значения длительности запускающих импульсов, когда схема работает как кипп-реле $\tau_{max} = ?\:\text{мкс}$, $\tau_{min} = 11,5\:\text{мкс}$.

Выяснили, что при амплитудах входного импульса, лежащего в диапазоне $0.325\:\text{В}<\tau<1,6\:\text{В}$, спусковая схема запускается и работает как кипп-реле.
\end{document}